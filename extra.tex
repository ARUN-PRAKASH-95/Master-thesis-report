\subsection{Elastic constitutive Equation and Elastic Compliance Tensor - By Hypothesis of Strain Equivalence}
\indent\indent\indent  Consider the damage effect tensor $\mathbb{M}$ defined by second-order symmetric damage tensor $D$ Applying hypothesis of strain equivalence and the effective stress in the matrix form to the elastic constitutive equation, we get
\begin{equation}
[\epsilon_{p}]  =  [S_{pq} (D)][\sigma_{q}]  =  [S_{pq}^{0}][\tilde{\sigma_{q}}]
\end{equation}
\begin{equation}
                =  [S_{pr}^{0}] [M_{rq} (D)][\sigma_{q}]
\end{equation}
Thus the elastic constitutive equation and the corresponding compliance matrix of a damaged material specified by the damage effect tensor are given by
\begin{equation}
[\epsilon_{p}]  =  [S_{pq} (D)][\sigma_{q}]
\end{equation}
\begin{equation}
[S_{pq} (D)]  = [S_{pq}^{0}][M_{rq} (D)]
\end{equation}
where $[S_{pr}^{0}]$  and $[M_{rq}^{1} (D)]$ are the compliance matrix of an undamaged elastic material and matrix of the damage effect tensor $\mathbb{M}$, respectively. The explicit representation of the compliance matrix is given as follows
$$
[S_{pq} (D)]  = [S_{pq}^{0}][M_{rq} (D)] =    
 \begin{bmatrix}
  \frac{1}{\tilde{E}_{1}} \; & \frac{-\tilde{\nu}_{21}}{\tilde{E}_{2}} \; & \frac{-\tilde{\nu}_{31}}{\tilde{E}_{3}} \; & 0 \; & 0\; & 0 \\
  \frac{-\tilde{\nu}_{12}}{\tilde{E}_{1}}\; & \frac{1}{\tilde{E}_{2}}\; & \frac{-\tilde{\nu}_{32}}{\tilde{E}_{3}}\; & 0\; & 0\; & 0 \\
  \frac{-\tilde{\nu}_{13}}{\tilde{E}_{1}}\; & \frac{-\tilde{\nu}_{23}}{\tilde{E}_{2}}\; & \frac{1}{\tilde{E}_{3}}\; & 0\; & 0\; & 0 \\
  0\; & 0\; & 0\;  & \frac{1}{2\tilde{G}_{23}}\; &0\; & 0 \\
  0\; & 0\; & 0\; & 0\; & \frac{1}{2\tilde{G}_{31}}\; & 0 \\
  0\; & 0\; & 0\; & 0\; & 0\; &\frac{1}{2\tilde{G}_{12}}  
 \end{bmatrix}
 $$
where  $\tilde{E}_{i}$,$\tilde{G}_{ij}$ and $\tilde{\nu}_{ij}$ denote the Young's modulus in i-direction, the shear modulus in i-j plane and Poissons's ration representing the transverse strain in j-direction induced by uniaxial stress in i-direction, respectively and are expressed as

\begin{equation}
\frac{1}{\tilde{E}_{i}} = \frac{1}{E}\Phi_{i}, \;\; \frac{1}{2\tilde{G}_{ij}}  = \frac{1}{2G}\Phi_{ij},\;\; \frac{\tilde{\nu}_{ij}}{\tilde{E}_{i}} =  \frac{\nu}{2E}\Phi_{i},
\end{equation} 
\begin{equation}
\Phi_{i}  =   (1  - D_{i})^{-1}, \;\; \Phi_{ij} =  \frac{1}{2}(\Phi_{i}  +  \Phi_{j}) 
\end{equation}
\begin{align*}
(no \; sum \; for \; i; \; i,j = 1,2,3)
\end{align*}
\section{Inelastic constitutive equation and Damage evolution equation with isotropic damage}
\indent\indent\indent  Material damage induced by microcracks in isotropic or random distribution can be characterized by isotropic damage. Even though the microcracks have apparent orientation in their distribution and geometry, their mechanical effect may be assumed to be isotropic if their size and density are sufficiently small. The damage in this case can be represented by a scalar damage variable D.
\subsection{Internal variables and Thermodynamic constitutive theory}
\indent\indent\indent  According to $principle\; of \;local state$, a non-equilibrium process in a continuum can be described as a succession of its equilibrium state, and hence by the process of the change in its state variables. Thus, the thermodynamic state of a damaging material is specified by total strain $\epsilon$ and the temperature $T$, in addition to the scalar damage variable $D$. Therefore the Helmholtz free energy function per unit mass is now given by
\begin{equation}
\Psi  = \Psi(\epsilon^{e},\; T,\; D)
\end{equation}
Substituting the above equation in Clausius-Duhem inequality gives,
\begin{equation}
(\sigma  -  \rho\frac{\partial\Psi}{\partial\epsilon^{e}}):\dot{\epsilon}^{e} - \rho(s +\frac{\partial\Psi}{\partial T})\dot{T}  - \rho\frac{\partial\Psi}{\partial D}\dot{D}  - \frac{gradT}{T}.q  \geq 0 
\end{equation}
In order that inequality (49) may be satisfied by any change of the elastic strain $\epsilon^{e}$ and the temperature $T$, the coefficients of the first and the second term of the inequality must be zero. From this requirement we have the elastic constitutive equation and thermal state equation
\begin{equation}
\sigma  = \rho\frac{\partial\Psi}{\partial\epsilon^{e}}, \;\;\;\;\;\;  s = \frac{\partial\Psi}{\partial T} 
\end{equation}
By the use of these relations, Eqn(49) results in
\begin{equation}
- \rho\frac{\partial\Psi}{\partial D}\dot{D}  - \frac{gradT}{T}.q  \geq 0 
\end{equation}
Therefore for the terms in the above inequality, we define associated variables with the the internals state variable $D$ and the heat flux $q$:
\begin{equation}
Y \equiv \rho\frac{\partial\Psi}{\partial D}  \;\;\;\;\;\; g \equiv gradT 
\end{equation}
Therefore the Clausius-Duhem inequality can be expressed eventually in the form,
\begin{equation}
\Phi  =  Y.\dot{D}  \;  + \; \frac{g}{T} .q  \geq 0,
\end{equation}
For the above dissipation process, we define generalized flux vector $J$ and a generalized force vector $X$ as follows:
\begin{equation}
J \equiv {\dot{D},\;\; q},
\end{equation}
\begin{equation}
X \equiv {Y,\;\; \frac{g}{T}}
\end{equation}
Thus Eqn(53) can be expressed in compact form,
\begin{equation}
\Phi\;\; = X\;.\;J \; \geq 0,
\end{equation}
When the dissipation is expressed $\Phi$ in the form of Eq. (56), the evolution equation for the generalized flux vector $J$ can be derived from a potential function F of the generalized force X. Then, we now postulate a dissipation potential function $F(X)$ in the form 
\begin{equation}
F(X) \; = \; F(Y, \; \frac{g}{T}; \; D, \; T),
\end{equation}
\begin{equation}
F = 0, \;\; for \;\; X = 0.
\end{equation}
The evolution equation of the flux vector $J$ is given by
\begin{equation}
J = \dot{\Lambda}\frac{\partial F}{\partial X},
\end{equation}
where $\dot{\Lambda}$ is an indeterminate scalar multiplier whose value is identified by the consistency condition. The explicit expression of Eqn. (59) are given as follows
\begin{equation}
\dot{D} \; = \; \dot{\Lambda}\frac{\partial F}{\partial Y}, \;\;\;\;\;\;  q\; = \; \dot{\Lambda}\frac{\partial F}{\partial (g/T)}
\end{equation}

\section{Constitutive damage model}
\indent\indent\indent  The thermodynamics of irreversible process is a general framework that can be used to formulate constitutive equations.  In order to establish a constitutive law, a scalar function corresponding to the complementary free energy density of the material has been defined. This function must be zero at the origin with respect to the free variables and must be positive definite. The proposed complementary free energy density for a damaged orthotropic lamina is given by

\begin{equation}
\psi = \frac{\sigma_{1}^{2}}{2(1 - d_{1})E_{1}} + \frac{\sigma_{2}^{2}}{2(1 - d_{2})E_{2}}  +  \frac{\nu_{12}}{E_{1}}\sigma_{1}\sigma_{2}  +  \frac{\sigma_{12}^{2}}{2(1 - d_{4})G_{12}} 
\end{equation}\\
where $E_{1}$,$G_{12}$, and $\nu_{12}$  are the Young's modulus, shear modulus and Poisson's ratio for the undamaged elastic material respectively. The damage variables $d_{1}$, $d_{2}$, and $d_{4}$ ensures that the composite ply maintains the original plane of material symmetry regardless of the damage state of the material. The variable $d_{1}$ is associated with the fibre failure, $d_{2}$  with the transverse matrix cracking and $d_{4}$ is influenced by longitudinal and transverse cracks. To ensure the thermodynamically irreversibility of the damage evolution, the rate of change of complementary free energy $\dot{G}$ minus the externally supplied work $\dot{\sigma}:\epsilon$, must not be negative:
\begin{equation}
\dot{G} - \dot{\sigma}:\epsilon  \geq 0
\end{equation}
This inequality describes the positiveness of the dissipated energy and has to be fulfilled by any constitutive model. Expanding the inequality in terms of damage variables and stress tensor gives
\begin{equation}
(\frac{\partial G}{\partial \sigma} - \epsilon) : \dot{\sigma} \; + \;\frac{\partial G}{\partial d_{1}}\dot{d_{1}}\; +\; \frac{\partial G}{\partial d_{2}}\dot{d_{2}}\; +\; \frac{\partial G}{\partial d_{6}}\dot{d_{6}}  \geq  0
\end{equation}
Since the stresses can vary freely, the expression in the parenthesis must be equal to zero to ensure the positive dissipation of mechanical energy. Therefore the strain tensor is first derivative of the complementary free energy density with respect to stress tensor.
\begin{equation}
\epsilon \; = \; \frac{\partial G}{\partial \sigma}  \; = \; H\;:\;\sigma
\end{equation}
Therefore the compliance tensor $H$ can be given as
$$
H \;=\;  \frac{\partial^{2} G}{\partial \sigma^{2}}\; =\;
 \begin{bmatrix}
  \frac{1}{(1 - d_{1})E_{1}}\; &\; -\frac{\nu_{21}}{E_{1}}\; &\; 0  \\
  \\
   -\frac{\nu_{12}}{E_{2}}\; &\; \frac{1}{(1 - d_{2})E_{2}}\; & \; 0  \\
   \\
  0\; &\; 0\; &\; \frac{1}{(1 - d_{6})G_{12}} 
 
 \end{bmatrix}
 $$
The complementary free energy density for a 3D case is given by
\begin{equation}
\psi \;=\; \frac{\sigma_{1}^{2}}{2(1 - d_{1})E_{1}} + \frac{\sigma_{2}^{2}}{2(1 - d_{2})E_{2}}\;  + \; \frac{\sigma_{3}^{2}}{2(1 - d_{3})E_{3}}\; + \; \frac{\nu_{12}}{E_{1}}\sigma_{1}\sigma_{2}\; +\; \frac{\nu_{13}}{E_{1}}\sigma_{1}\sigma_{3} \; +\; 
\end{equation}
\begin{align*}
\frac{\nu_{23}}{E_{2}}\sigma_{2}\sigma_{3}\; + \; \frac{\sigma_{12}^{2}}{2(1 - d_{4})G_{12}}\; + \; \frac{\sigma_{23}^{2}}{2(1 - d_{5})G_{23}}\; +  \; \frac{\sigma_{13}^{2}}{2(1 - d_{6})G_{13}}
\end{align*}
\newpage
The compliance tensor $H$ for 3D case can be given as
$$
H \; = \; 
 \begin{bmatrix}
  \frac{1}{(1 - d_{1})E_{1}} & -\frac{\nu_{12}}{E_{1}} & -\frac{\nu_{13}}{E_{1}} & 0 & 0 & 0 \\
  \\
  -\frac{\nu_{21}}{E_{2}} & \frac{1}{(1 - d_{2})E_{2}} & -\frac{\nu_{23}}{E_{2}} & 0 & 0 & 0 \\
	\\  
  -\frac{\nu_{31}}{E_{3}} & -\frac{\nu_{32}}{E_{3}} & \frac{1}{(1 - d_{3})E_{3}} & 0 & 0 & 0 \\
  \\
  0 & 0 & 0 & \frac{1}{(1 - d_{4})G_{12}} & 0 & 0 \\
  \\
  0 & 0 & 0 & 0 & \frac{1}{(1 - d_{5})G_{23}} & 0 \\
  \\
  0& 0 & 0 & 0 & 0 & \frac{1}{(1 - d_{5})G_{13}} 
 \end{bmatrix}
 $$
\subsection{Damage activation functions}
\indent\indent\indent Damage activation function(DAF) defines the elastic domain with in which the material is linearly elastic. The determination of the domain of elastic response under stress states is an essential component of accurate damage model. The DAF can be defined as
\begin{equation}
g_{m} \; = \; \hat{g_{m}} \; - \; \hat{\gamma_{m}}  \leq  0
\end{equation} 
where $\hat{g_{m}}$ is the positive loading function that depends on stress states and $\hat{\gamma_{m}}$ is the updated threshold function and $m$ refers to the failure mode.


\section{UMAT implementation}

\begin{center}
\begin{tikzpicture}[node distance = 2.5cm]

\node(straininc) [startstop] {Apply increment $\Delta\epsilon$};
\node(strainadd) [process, below of = straininc] {$\epsilon_{n+1} = \epsilon_{n} + \Delta\epsilon $};
\node(Failure)   [process, below of = strainadd] {Failure Criteria};
\node(Damageon)  [decision, below of = Failure, yshift = -0.5cm] {Damage onset?};
\node(Damagecalc) [process, below of = Damageon, yshift = -0.5cm] {Calculate damage variables};
\node(Materialeqn)[process, below of = Damagecalc] {Material constitutive equations};
\node(Tangent)[process, below of = Materialeqn] {Updating stress: $\sigma = C_{d}.\; \epsilon_{n + 1}$ \; Updating     $\frac{\partial \Delta \sigma}{\partial \Delta \epsilon}$};
\node(abaqus)[startstop, below of = Tangent] {ABAQUS solver};
\node(Converge)[decision, below of = abaqus, yshift = -0.5cm] {Converged?};

\draw [arrow] (straininc) -- (strainadd);
\draw [arrow] (strainadd) -- (Failure);
\draw [arrow] (Failure) -- (Damageon);
\draw [arrow] (Damageon) -- (Damagecalc);
\draw [arrow] (Damagecalc) -- (Materialeqn);
\draw [arrow] (Materialeqn) -- (Tangent);
\draw [arrow] (Tangent) -- (abaqus);
\draw [arrow] (abaqus) -- (Converge);
\draw [arrow] (Converge.west) -| ++(-4,2) -- ++(0,17.5) -- ++(0,2) --                
     node[xshift=-2.3cm,yshift=-6.5cm, text width=3.2cm]
     {Next increment until final failure or end of analysis}node[xshift=1.3cm,yshift=-22cm, text width=3.2cm]{Yes}(straininc.west);
\draw [arrow] (Converge.east) -| ++(4,2) -- ++(0,17.5) -- ++(0,1) --(0,-1)                
     node[near start,xshift=1.8cm,yshift=-7cm, text width=3.2cm]
     {Iterate until convergence}node[xshift=5cm,yshift=-21cm, text width=3.2cm]{No};
\draw [arrow] (0,-17.5)  -| ++(-3.5,0.7) -- ++(0,14.5) -- ++(0,1) -- (0,-1.3)               
     node[xshift=-3.5cm,yshift=-7cm, text width=3.2cm]
     {Do for each integration point};     
     
     

\end{tikzpicture}
\end{center}

\newpage
\section{Three dimensional Continuum damage mechanics model}
\indent\indent\indent A symmetric second order tensor D, has been chosen as the damage tensor and its principal directions are assumed to coincide with the principal material directions. The $i^{th}$ eigen value $d_{i}$ represents the effective fractional reduction in load carrying area on planes that are perpendicular to the $i^{th}$ principal material direction. The first principal direction coincides with the fiber direction and the second and the third material direction are made to coincide with the transverse and ply stacking direction. The eigen value of the damage tensor must be in the range $0 \leq d_{i} \leq  1  $, where $d_{i}$ = 0 means complete lack of microcracks in the $i^{th}$ principal material direction, while $d_{i}$ = 1 means complete separation of material across the planes that are normal to the $i^{th}$ principal material direction.

$$
D \; = \; 
 \begin{bmatrix}
  d_{1} & 0 & 0  \\
  \\
  0 & d_{2} & 0  \\
  \\  
  0 & 0 & d_{3} \\
  
 \end{bmatrix}
 $$  
 
The effective stress tensor can be defined as 
\begin{equation}
   \bar{\sigma} \; = \; M(D) : \sigma
\end{equation}  
where
\begin{equation*}
M(D) \; = \; diag\;[1/\omega_{11}\;\;1/\omega_{22}\;\;1/\omega_{33}\;\;1/\omega_{12}\;\;1/\omega_{13}\;\;1/\omega_{23}\;]
\end{equation*}
\begin{equation*}
\omega_{11}\; = \; 1 - d_{1}, \;\; \omega_{22}\; = \; 1 - d_{2}, \;\;\omega_{33}\; = \; 1 - d_{3} \;\; \omega_{12}\; = \; \sqrt{(1 - d_{1})(1 - d_{2})} \;\;
\end{equation*}
\begin{equation*}
 \omega_{13}\; = \; \sqrt{(1 - d_{1})(1 - d_{3})} \;\;  \omega_{23}\; = \; \sqrt{(1 - d_{2})(1 - d_{3})} \;\;
\end{equation*}
The effective stress is introduced into the strain energy equivalence principle. The strain energy of the material without damage can be expressed as 
\begin{equation}
W = \frac{1}{2}\sigma:C^{-1}:\sigma
\end{equation}
The equivalence of strain energy for damaged material is given by
\begin{equation}
W^{d} = \frac{1}{2}\bar{\sigma}:C^{-1}:\bar{\sigma} \; = \; \frac{1}{2}\sigma: M^{T}:C^{-1}:M:\sigma
\end{equation}
So the strain can be calculated as follows
\begin{equation}
\epsilon = \frac{\partial W^{d}}{\partial \sigma} = (M^{T}:C^{-1}:M):\sigma
\end{equation}
The constitutive equation of the damaged composite laminates is given as 
\begin{equation}
C^{d}\; = \; M^{-1}:C:M^{T,-1}
\end{equation}
\newpage
The stiffness matrix of the damaged composite lamina in matrix form can be represented as follows
\begin{tiny}
\begin{equation*}
C^{d} \; = \; 
 \begin{bmatrix}
  C_{11}(1 - d_{1})^{2} & C_{12}(1 - d_{1})(1 - d_{2}) & C_{13}(1 - d_{1})(1 - d_{3})  & 0 & 0 & 0 \\
  \\
  C_{21}(1 - d_{2})(1 - d_{1}) & C_{22}(1 - d_{2})^{2}  & C_{23}(1 - d_{2})(1 - d_{3}) & 0 & 0 & 0 \\
 \\  
  C_{31}(1 - d_{3})(1 - d_{1}) & C_{32}(1 - d_{3})(1 - d_{2}) & C_{33}(1 - d_{3})^{2}  & 0 & 0 & 0 \\
  \\
  0 & 0 & 0 & C_{44}(1 - d_{1})(1 - d_{2})   & 0 & 0 \\
  \\
  0 & 0 & 0 & 0 & C_{55}(1 - d_{1})(1 - d_{3}) & 0 \\
  \\
  0& 0 & 0 & 0 & 0 & C_{66}(1 - d_{2})(1 - d_{3}) 
 \end{bmatrix}
\end{equation*}
\end{tiny}
\subsection{Damage initiation criteria}
\indent\indent\indent   The failure modes considered here include fiber tension or compression failure, matrix tension or compression failure,delamination. Since tension and compression failure in each direction cannot happen at the same integration  point at same time, three failure modes, $F_{f}, F_{m}, F_{z} $ are used for the failure modes in three principal directions. The failure criteria utilized here are in the form of 3D hashin criteria in quadratic strain
\begin{equation}
F_{f}^{2} =  
	\begin{cases}
	\Big(\frac{\epsilon_{11}}{\epsilon_{11}^{f,t}}\Big)^{2} \; + \; \Big(\frac{\epsilon_{12}}{\epsilon_{12}^{f}}\Big)^{2} \; + \; \Big(\frac{\epsilon_{13}}{\epsilon_{13}^{f}}\Big)^{2} \; \geq  \; 0  \; \; \; \; \;  (\epsilon_{11}  >  0)  \\
	\\
	\Big(\frac{\epsilon_{11}}{\epsilon_{11}^{f,c}}\Big)^{2}  \; \geq  \; 0 \; \; \; \; \; \; \;  \; \; \; \;  (\epsilon_{11}  <  0) 
	
	\end{cases}
\end{equation}

\begin{equation}
F_{m}^{2} =  
	\begin{cases}
	
	\frac{(\epsilon_{22} + \epsilon_{33} )^{2}}{\epsilon_{22}^{f,t} \epsilon_{33}^{f,t}} \;  - \; \frac{\epsilon_{22}\epsilon_{33}}{(\epsilon_{23}^{f})^{2}} \; + \; \Big(\frac{\epsilon_{12}}{\epsilon_{12}^{f}}\Big)^{2} \; + \; \Big(\frac{\epsilon_{13}}{\epsilon_{13}^{f}}\Big)^{2} \;  + \; \Big(\frac{\epsilon_{23}}{\epsilon_{23}^{f}}\Big)^{2} \;\geq  \; 0 \; \; \; \; \;  (\epsilon_{22}  >  0) \\
	\\
	
	\frac{(\epsilon_{22} + \epsilon_{33} )^{2}}{\epsilon_{22}^{f,c} \epsilon_{33}^{f,c}} \; +  \;  \frac{\epsilon_{22} + \epsilon_{33}}{\epsilon_{22}^{f,c}}\Big(\frac{\epsilon_{22}^{f,c}}{2\epsilon_{12}^{f}} \; - \; 1\Big) \;  - \; \frac{\epsilon_{22}\epsilon_{33}}{(\epsilon_{23}^{f})^{2}} \; + \; \Big(\frac{\epsilon_{12}}{\epsilon_{12}^{f}}\Big)^{2} \; + \; \Big(\frac{\epsilon_{13}}{\epsilon_{13}^{f}}\Big)^{2} \;  + \; \Big(\frac{\epsilon_{23}}{\epsilon_{23}^{f}}\Big)^{2} \;\geq \; 0 \; \; \; \; \; \;  (\epsilon_{22}  >  0) 
	 
	
	\end{cases}
\end{equation}


\begin{equation}
F_{z}^{2} =  
	\begin{cases}

	\Big(\frac{\epsilon_{33}}{\epsilon_{33}^{f,t}}\Big)^{2} \; + \; \Big(\frac{\epsilon_{13}}{\epsilon_{13}^{f}}\Big)^{2} \; + \; \Big(\frac{\epsilon_{23}}{\epsilon_{23}^{f}}\Big)^{2} \; \geq  \; 0 \; \; \; \; \;  (\epsilon_{33}  >  0) \\

\\
	\Big(\frac{\epsilon_{33}}{\epsilon_{33}^{f,c}}\Big)^{2} \; + \; \Big(\frac{\epsilon_{13}}{\epsilon_{13}^{f}}\Big)^{2} \; + \; \Big(\frac{\epsilon_{23}}{\epsilon_{23}^{f}}\Big)^{2} \; \geq  \; 0 \; \; \; \; \;  (\epsilon_{33}  >  0) \\



	\end{cases}
\end{equation}
\\
\\
in which \; $\epsilon_{ii}^{f,t}  =   \frac{\sigma_{i}^{f,t}}{C_{ii}}$, \; $\epsilon_{ii}^{f,c}  =   \frac{\sigma_{i}^{f,c}}{C_{ii}} \;\;\; (i = 1,2,3)$, \; $\epsilon_{12}^{f}  =   \frac{\sigma_{12}^{f}}{C_{44}}$,  \; $\epsilon_{13}^{f}  =  \frac{\sigma_{13}^{f}}{C_{55}}$, \; $\epsilon_{23}^{f}  =   \frac{\sigma_{23}^{f}}{C_{66}}$,

\subsection{Damage evolution law}
\indent\indent\indent  If the failure is detected at a material point, the properties of the material must be degraded according to a material property degradation model. A number of post-failure material property degradation model has been proposed for progressive failure analysis. Most of these material degradation models belong to on of three general categories: instantaneous unloading, gradual unloading, or constant stress at failure material point. The most simple damage model for laminated composites is the ply discount method in which the material stiffness is removed completely when the failure is detected at material point. This method can be used to get rough estimate of the final failure of a composite structure. However, continuum damage mechanics is a more accurate methodology, in which gradual unloading of a ply after the onset of damage is simulated by means of a material degradation model. The standard implementation of strain-softening constitutive models results in mesh-dependent results i.e., the solution is non-objective with respect to the mesh refinement and the computed dissipated energy decreases with the reduction of the element size. So the fracture energy is introduced into the damage evolution law to reduce mesh sensitivity. The damage evolution equations are given below
\\
\\
In fibre direction,
\begin{equation}
d_{1} = 1 - \frac{e^{(-\sigma_{11}^{f}\epsilon_{11}^{f}L^{c}(F_{f} - 1))    /G_{c,1}}}{F_{f}}   
\end{equation}
\\
In matrix direction,
\begin{equation}
d_{2} = 1 - \frac{e^{(-\sigma_{22}^{f}\epsilon_{22}^{f}L^{c}(F_{m} - 1))    /G_{c,2}}}{F_{m}}   
\end{equation}
\\
In ply stacking direction,
\begin{equation}
d_{3} = 1 - \frac{e^{(-\sigma_{33}^{f}\epsilon_{33}^{f}L^{c}(F_{z} - 1))    /G_{c,3}}}{F_{z}}   
\end{equation}
\\