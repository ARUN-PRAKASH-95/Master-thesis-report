\documentclass[a4paper,14pt]{extarticle}
    \usepackage{amsmath}
    \usepackage{amsfonts}
    \usepackage[utf8]{inputenc}
    \usepackage{titlepic}
    \usepackage{stackengine}
    \usepackage{lipsum}
    \usepackage{graphicx}
    \usepackage{booktabs}
    \usepackage{cite}
	\usepackage{caption}
	\usepackage{float}
	\usepackage{subcaption}
    \usepackage{multirow}
    \usepackage[top=1in, bottom=1in, left=1in, right=1in]{geometry}
    \usepackage{fancyhdr}
    \usepackage{relsize}
    \usepackage{tikz}
    \usetikzlibrary{shapes.geometric,arrows}
	\makeatletter
    \setlength{\@fptop}{0pt}
    \makeatother
    \tikzstyle{startstop} = [rectangle, rounded corners, minimum width = 4cm, minimum height=1cm, text centered, draw = black, fill=blue!10]    
     \tikzstyle{process} = [rectangle,  minimum width = 4cm, minimum height=1cm, text centered, text width = 4cm,draw = black, fill = blue!10]  
	 \tikzstyle{decision} = [diamond, minimum width = 4cm, minimum height=1cm, text centered, draw = black, fill = blue!10]  
     \tikzstyle{arrow} = [thick,->,>=stealth]  


\usepackage{titling}
\pretitle{\begin{center}\fontsize{18bp}{18bp}\selectfont}
    \posttitle{\vspace{25bp}\par\includegraphics[width=60mm]{logo.png}\par\end{center}}
\preauthor{\begin{center}\fontsize{14bp}{14bp}\selectfont}
    \postauthor{\par\end{center}}
\predate{\begin{center}}
    \postdate{\par\end{center}}  

\title{\LARGE{\textbf{Development and implementation of material subroutine for fibre reinforced plastics in a commercial FEM software}}\\[0.7cm]\smaller \textbf{Master thesis \\[0.5cm] Winter semester 2018/19}\vspace*{1cm}}


\author{
\vspace{2cm}
\large{Presented by: Arun prakash Ganapathy}
\\[0.4cm]
\large{Supervised by: Dr.-Ing Dominik Laveuve}
\\[0.4cm]
\large{Mat.Nr: 63876}
\\[0.4cm]
\large{E-mail: arun-prakash.ganapathy@student.tu-freiberg.de}}
\date{}

\begin{document}
\maketitle
\newpage
\section{Introduction}
\subsection{Background and motivation}
\indent\indent Composite materials are made of two or more dissimilar materials of different physical or chemical properties when combined, create a material with properties unlike the individual constituent materials. The earliest use of composite materials date back to 3400 B.C when Mesopotamians glued wood strips at different angles to create plywood. Another notable example is the bow made by Mongols during 1200 A.D, which is made from a combination of bamboo, wood, cattle tendons and silk bonded with natural resin. In the early, 1900s bakelite based composites were developed for its non-conductivity and heat resistant properties and widely used in industrial and consumer applications. Now a days advanced composite materials are widely used in structural design in various industries such as aerospace, automobile, marine, petrochemical etc., due to their superior properties over traditional engineering materials.\\
\end{document}


\subsubsection{Maximum strain criteria}
\indent\indent\indent The maximum strain criterion just checks whether the strain in the given material direction exceeds the failure strain or not. The maximum strain criterion for each material direction is given below
\\
\\
\begin{table}[h!]
  \begin{center}
     \begin{tabular}{l  c  c} 
     \hline
     \\
      \textbf{Damage direction} \;\;& \textbf{Tension} \;& \textbf{Compression}\\
      \\
      \hline
      \\
      Fibre direction 1 ($F_{f}$) & \Large{$\frac{\epsilon_{1}}{\epsilon_{t1}} $}\small{ $\leq 1$} &  \Large{$\frac{\epsilon_{1}}{\epsilon_{c1}} $}\small{ $\leq 1$}\\
      \\
      Matrix direction 2 ($F_{m}$)  &  \Large{$\frac{\epsilon_{2}}{\epsilon_{t2}} $}\small{$\leq 1$}  & \Large{$\frac{\epsilon_{2}}{\epsilon_{c2}} $}\small{$\leq 1$}\\
      \\
      Matrix direction 3 ($F_{z}$) &  \Large{$\frac{\epsilon_{3}}{\epsilon_{t3}} $}\small{$\leq 1$}  &   \Large{$\frac{\epsilon_{2}}{\epsilon_{c3}} $}\small{$\leq 1$}\\
       \\
       \hline
    \end{tabular}
    \\
    \caption{Maximum strain criterion}
    \label{tab:Maximum strain criterion}
  \end{center}
\end{table}\\
where $\epsilon_{ti}$ and $\epsilon_{ci}$ are failure strains in tension and compression respectively, in each material direction. 
\\
\subsubsection{Maximum stress criteria}
\indent\indent\indent Since the load-carrying area decreases due to increase in damage, the effective of stress gets magnified in the damaged material. Therefore, in case of maximum stress criteria normal stress components of the effective stress are checked against the failure strength in each principal material direction. The effective stress can computed using the Eqn.(\ref{eqn:effective_stress_tensor}). The maximum stress criteria for each material direction is given in the table below
\\
\\
\begin{table}[h!]
  \begin{center}
     \begin{tabular}{l  c  c} 
     \hline
     \\
      \textbf{Damage direction} \;\;& \textbf{Tension} \;& \textbf{Compression}\\
      \\
      \hline
      \\
      Fibre direction 1 ($F_{f}$) & \Large{$\frac{\tilde{\sigma}_{11}}{X_{T}} $}\small{ $\leq 1$} & \Large{$\frac{\tilde{\sigma}_{11}}{-X_{C}} $}\small{ $\leq 1$} \\
      \\
      Matrix direction 2 ($F_{m}$)  &  \Large{$\frac{\tilde{\sigma}_{22}}{Y_{T}} $}\small{ $\leq 1$}  & \Large{$\frac{\tilde{\sigma}_{22}}{-Y_{C}} $}\small{ $\leq 1$}\\
      \\
      Matrix direction 3 ($F_{z}$) &  \Large{$\frac{\tilde{\sigma}_{33}}{Z_{T}} $}\small{ $\leq 1$}  &   \Large{$\frac{\tilde{\sigma}_{33}}{-Z_{C}} $}\small{ $\leq 1$}\\
       \\
       \hline
    \end{tabular}
    \\
    \caption{Maximum stress criterion}
    \label{tab:Maximum stress criterion}
  \end{center}
\end{table}\\
\\
where $X_{T}$,$ Y_{T} $ and $Z_{T}$ are failure strength in tension and $X_{C}$,$ Y_{C} $ and $Z_{C}$ are failure strength in compression in each principal material direction respectively.

\subsubsection{3D Hashin's quadratic strain criteria}
\indent\indent\indent Hashin's quadratic strain criteria includes shear strain components in addition to the normal strain components. The criteria for each material direction is given below\\
\\
\\
Fibre direction 1,
\\
\begin{equation}
\large{F_{f}^{2} =  
	\begin{cases}
	\Big(\frac{\epsilon_{11}}{\epsilon_{11}^{f,t}}\Big)^{2} \; + \; \Big(\frac{\epsilon_{12}}{\epsilon_{12}^{f}}\Big)^{2} \; + \; \Big(\frac{\epsilon_{13}}{\epsilon_{13}^{f}}\Big)^{2} \; \geq  \; 1  \; \; \; \; \;  (\epsilon_{11}  >  0)  \\
	\\
	\Big(\frac{\epsilon_{11}}{\epsilon_{11}^{f,c}}\Big)^{2}  \; \geq  \; 1 \; \; \; \; \; \; \;  \; \; \; \;  (\epsilon_{11}  <  0) 
	
	\end{cases}}
\end{equation}
\\
\\
Matrix direction 2,
\\
\begin{equation}
\large{F_{m}^{2} =  
	\begin{cases}
	
	\frac{(\epsilon_{22} + \epsilon_{33} )^{2}}{\epsilon_{22}^{f,t} \epsilon_{33}^{f,t}}   -  \frac{\epsilon_{22}\epsilon_{33}}{(\epsilon_{23}^{f})^{2}}  +  \Big(\frac{\epsilon_{12}}{\epsilon_{12}^{f}}\Big)^{2}  + \Big(\frac{\epsilon_{13}}{\epsilon_{13}^{f}}\Big)^{2}  +  \Big(\frac{\epsilon_{23}}{\epsilon_{23}^{f}}\Big)^{2} \;\geq  \; 1 \; \; \; \; \;  (\epsilon_{22}  >  0) \\
	\\
	
	\frac{(\epsilon_{22} + \epsilon_{33} )^{2}}{\epsilon_{22}^{f,c} \epsilon_{33}^{f,c}}  +  \frac{\epsilon_{22} + \epsilon_{33}}{\epsilon_{22}^{f,c}}\Big(\frac{\epsilon_{22}^{f,c}}{2\epsilon_{12}^{f}}  -  1\Big)   -  \frac{\epsilon_{22}\epsilon_{33}}{(\epsilon_{23}^{f})^{2}}  +  \Big(\frac{\epsilon_{12}}{\epsilon_{12}^{f}}\Big)^{2}\; \; \; \; \; \; \; \; \; \; \; \;  (\epsilon_{22}  >  0) \\ 
\\	
	\; \; \; \; \; \; \; \;\; \; \; \; \; \; \; \;\; \; \; \; \; \; \; \;  \; \;\; \; \; \; \; \; \; \;\; \; \; \; \;  + \Big(\frac{\epsilon_{13}}{\epsilon_{13}^{f}}\Big)^{2}   +  \Big(\frac{\epsilon_{23}}{\epsilon_{23}^{f}}\Big)^{2} \;\geq \; 1 
	
	
	\end{cases}}
\end{equation}

Matrix direction 3,
\\
\begin{equation}
\large{F_{z}^{2} =  
	\begin{cases}

	\Big(\frac{\epsilon_{33}}{\epsilon_{33}^{f,t}}\Big)^{2} \; + \; \Big(\frac{\epsilon_{13}}{\epsilon_{13}^{f}}\Big)^{2} \; + \; \Big(\frac{\epsilon_{23}}{\epsilon_{23}^{f}}\Big)^{2} \; \geq  \; 1 \; \; \; \; \;  (\epsilon_{33}  >  0) \\

\\
	\Big(\frac{\epsilon_{33}}{\epsilon_{33}^{f,c}}\Big)^{2} \; + \; \Big(\frac{\epsilon_{13}}{\epsilon_{13}^{f}}\Big)^{2} \; + \; \Big(\frac{\epsilon_{23}}{\epsilon_{23}^{f}}\Big)^{2} \; \geq  \; 1 \; \; \; \; \;  (\epsilon_{33}  >  0) \\



	\end{cases}}
\end{equation}
\\
\\
in which \large($\epsilon_{ii}^{f,t}=\frac{\sigma_{i}^{f,t}}{C_{ii}}$,  $\epsilon_{ii}^{f,c}  =   \frac{\sigma_{i}^{f,c}}{C_{ii}} \; (i = 1,2,3)$,  $\epsilon_{12}^{f}  =   \frac{\sigma_{12}^{f}}{C_{44}}$,   $\epsilon_{13}^{f}  =  \frac{\sigma_{13}^{f}}{C_{55}}$,  $\epsilon_{23}^{f}  =   \frac{\sigma_{23}^{f}}{C_{66}}$)

\subsection{Types of damage evolution}
\indent\indent\indent Once damage is initiated in a material point of composite material, the material property must be degraded. This results in strain-softening of the composite materials rather than strain hardening which is observed in conventional materials like metals. A number of post-damage models are proposed for progressive failure analysis and most of them belong to one of the following categories: instantaneous unloading, gradual loading, or constant stress at failure material point as shown in Fig.(\ref{fig:Types_of_damage_evolution}) \\  
\indent\indent\indent Since continuum damage mechanics is a more accurate methodology to predict the failure behaviour of composites, non-linear gradual unloading of the composite material is adopted and simulated in this work.  Therefore non-linear material properties degradation model is implemented in this work. Based on the type of damage variable chosen i.e., scalar, vector or 2nd order tensor the damage modelling can be classified into isotropic or anisotropic damage modelling. A brief description about the types of damage modelling, damage evolution equation used and material tangent stiffness etc., is given below
\begin{figure}[htbp]
\begin{center}
\includegraphics[width=0.7\textwidth]{{8.Types_of_damage_evolution}}
 \caption{Types of degradation behaviour in damaged composite materials}
 \label{fig:Types_of_damage_evolution}
 \end{center}
\end{figure}


\subsubsection{Isotropic damage}
\indent\indent\indent In case of isotropic distribution of cracks, the damage state is usually considered to be isotropic and only a scalar variable D is required to represent the damage state of the material. In this work, a damage evolution law of following form is considered for calculating damage, once the damage has initiated.\\
\\
\begin{equation}
  \large{ D \; = \; 1 - e^{-P(\epsilon - \epsilon_{f})}}
  \label{eqn: isotropic_damage}
\end{equation} 
\\
where $P$ is the softening parameter which determines slope of the damage evolution, $\epsilon$ is 1D strain value and $\epsilon_{f}$ is the failure strain. Since the Eqn.(\ref{eqn: isotropic_damage}) is an exponential equation, the damage D evolves exponentially from 0 to 1 and the strain-softening will be an exponential decay function. Once the damage starts to evolve the material property must be degraded. Therefore the material stiffness matrix of the  damaged material is given by,\\
\begin{equation}
\large{\mathbb{C}(D) \; = \; (1  - D) \mathbb{C}_{0} }
\end{equation} 
where $\mathbb{C}_{0}$ is the material stiffness matrix of the undamaged material. Therefore the stress-strain relation for a strain softening model is given by,\\
\begin{equation}
\large{\sigma \; = \; \mathbb{C}(D) : \epsilon }  
\end{equation}
\\
The finite element equations obtained for strain-softening model is non-linear and therefore Newton-Raphson technique is used to solve the resulting system of non-linear equations. To ensure the robustness of the Newton-Raphson method, it is important to compute the material tangent constitutive tensor $\mathbb{C}_{T}$. It can be derived as follows,\\
\begin{equation*}
\large{ \mathbb{C}_{T}  \; = \;\frac{\partial \sigma}{\partial \epsilon}  }
\end{equation*}

\begin{equation*}
\large{\sigma  \; = \; f(\epsilon, D) }
\end{equation*}

\begin{equation}
\large{\mathbb{C}_{T}  \; = \; \mathbb{C}(D) + \Big(-\mathbb{C}_{0} : \epsilon \otimes \frac{\partial D}{\partial \epsilon }\Big)    }
\end{equation}
\\
\subsubsection{Anisotropic damage}
\indent\indent\indent In case of orthotropic materials, the material property changes in mutually orthogonal direction. Therefore it is more appropriate to chose a second order damage tensor. In this work, a symmetric second order tensor $\underline{D}$ is chosen , whose principal directions are assumed to coincide with the principal material directions. The eigen values of the damage tensor $\underline{D}$ have a simple physical interpretation i.e., the $i^{th}$ eigen value $d_{i}$ represents the effective fractional reduction in load carrying area on planes that are perpendicular to $i^{th}$ principal material direction. The damage tensor $\underline{D}$ can be represented as,\\
$$
\underline{D} \; = \; 
 \begin{bmatrix}
  d_{1}  \;& 0  \; & 0  \\
  \\
  0 \; & d_{2} \; & 0  \\
  \\  
  0 \; & 0 \; & d_{3} \\
  
 \end{bmatrix}
 $$  
\\
The relationship between damaged and undamaged material can be established in different way which helps us derive the constitutive equation (i.e.,damaged elastic stiffness matrix) for the damaged material. In this work, hypothesis of strain energy equivalence (Section.\ref{Hypothesis of strain energy equivalence}) and hypothesis of strain equivalence (Section.\ref{Hypothesis of Strain Equivalence}) are used to derive the elastic stiffness matrix of the damaged material. According to hypothesis of strain energy equivalence, from Eqn.(\ref{eqn: S_HSEeq}) the elastic stiffness matrix of the damaged material can be derived as,\\
\begin{equation}
\mathbb{C}(D) \; = \; \mathbb{M}^{-1} : \mathbb{C}_{0} :  \mathbb{M}^{T,-1}  
\label{Damaged_elasticity_matrix_1}
\end{equation}
where $\mathbb{M}$ is the damage effect tensor from Section.(\ref{Matrix Representation of Damage effect tensors}). The matrix representation of the Eqn.(\ref{Damaged_elasticity_matrix_1}) is given as,
\begin{tiny}

\begin{equation*}
 C^{d} \; = \; 
 \begin{bmatrix}
  C_{11}(1 - d_{1})^{2} & C_{12}(1 - d_{1})(1 - d_{2}) & C_{13}(1 - d_{1})(1 - d_{3})  & 0 & 0 & 0 \\
  \\
  C_{21}(1 - d_{2})(1 - d_{1}) & C_{22}(1 - d_{2})^{2}  & C_{23}(1 - d_{2})(1 - d_{3}) & 0 & 0 & 0 \\
 \\  
  C_{31}(1 - d_{3})(1 - d_{1}) & C_{32}(1 - d_{3})(1 - d_{2}) & C_{33}(1 - d_{3})^{2}  & 0 & 0 & 0 \\
  \\
  0 & 0 & 0 & C_{44}(1 - d_{1})(1 - d_{2})   & 0 & 0 \\
  \\
  0 & 0 & 0 & 0 & C_{55}(1 - d_{2})(1 - d_{3}) & 0 \\
  \\
  0& 0 & 0 & 0 & 0 & C_{66}(1 - d_{1})(1 - d_{3}) 
 \end{bmatrix}
\end{equation*}

\end{tiny}

 According to hypothesis of strain equivalence, from Eqn.(\ref{eqn: S_HSeq}) the elastic stiffness matrix of the damaged material can be derived as,\\
 \begin{equation}
\mathbb{C}(D) \; = \; \mathbb{M}^{-1} : \mathbb{C}_{0}   
\label{Damaged_elasticity_matrix_2}
\end{equation}
The matrix representation of the Eqn.(\ref{Damaged_elasticity_matrix_2}) is given as,
\begin{tiny}

\begin{equation*}
 C^{d} \; = \; 
 \begin{bmatrix}
  C_{11}(1 - d_{1}) & C_{12}(1 - d_{1}) & C_{13}(1 - d_{1})  & 0 & 0 & 0 \\
  \\
  C_{21}(1 - d_{2}) & C_{22}(1 - d_{2})  & C_{23}(1 - d_{2}) & 0 & 0 & 0 \\
 \\  
  C_{31}(1 - d_{3}) & C_{32}(1 - d_{3})) & C_{33}(1 - d_{3})  & 0 & 0 & 0 \\
  \\
  0 & 0 & 0 & C_{44} \sqrt{(1 - d_{1})(1 - d_{2})}   & 0 & 0 \\
  \\
  0 & 0 & 0 & 0 & C_{55}\sqrt{(1 - d_{2})(1 - d_{3})} & 0 \\
  \\
  0& 0 & 0 & 0 & 0 & C_{66}\sqrt{(1 - d_{1})(1 - d_{3})}
 \end{bmatrix}
\end{equation*}

\end{tiny}

The standard implementation of strain-softening constitutive models results in mesh-dependent results i.e., the solution is non-objective with respect to the mesh refinement and the computed dissipated energy decreases with the reduction of the element size. So the fracture energy is introduced into the damage evolution law to reduce mesh sensitivity. The damage evolution equations are given below
\\
\\
In fibre direction 1,
\begin{equation}
\large{ d_{1} = 1 - \frac{e^{(-\sigma_{11}^{f}\epsilon_{11}^{f}L^{c}(F_{f} - 1))    /G_{c,1}}}{F_{f}} }   
\end{equation}
\\
In matrix direction 2,
\begin{equation}
\large{ d_{2} = 1 - \frac{e^{(-\sigma_{22}^{f}\epsilon_{22}^{f}L^{c}(F_{m} - 1))    /G_{c,2}}}{F_{m}} }  
\end{equation}
\\
In matrix direction 3,
\begin{equation}
\large{ d_{3} = 1 - \frac{e^{(-\sigma_{33}^{f}\epsilon_{33}^{f}L^{c}(F_{z} - 1))    /G_{c,3}}}{F_{z}} }  
\end{equation}
\\